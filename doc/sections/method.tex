\documentclass[doc/report.tex]{subfiles}

% Describe the method which you have chosen for solving the task and explain
% why you have chosen this particular method. 

\begin{document}

\section{Method}
In order to find a good method, multiple methods and approaches were tested
empirically. All of the tested methods are described in this chapter.


\subsection{Preprocessing}
% TODO

\subsubsection{Data partitioning}
As the given dataset was already split into training and test sets, no further
partitioning was performed. The training set was used for training and the test
set was used for evaluation of the classifier. The number of each class from
each dataset is listed in table \ref{tbl:datasets}.

\begin{table}[h]
    \centering
    \begin{tabular}{l|rr|r}
        class & training & test & all \\\hline
        airplane    & 484   & 243   & 727 \\
        car         & 645   & 323   & 968 \\
        cat         & 590   & 295   & 885 \\
        dog         & 468   & 234   & 702 \\
        flower      & 562   & 281   & 843 \\
        fruit       & 667   & 333   & 1000 \\
        motorbike   & 525   & 263   & 788 \\
        person      & 658   & 328   & 986 \\\hline
        all         & 4599  & 2300  & 6899
    \end{tabular}
    \caption{Distribution of classes in datasets.}
    \label{tbl:datasets}
\end{table}

\subsubsection{Image augmentation}
% TODO


\subsection{Feature Extraction}
% TODO

\subsubsection{Raw pixels}
One approach that was tested was to simply use raw pixel data as features. It
was done by converting the images to grayscale, resizing them to a fixed size
and concatenating all of the pixel intensities to a feature vector. The images
were resized to a size of 100x100 pixels resulting in 10000 features.

\subsubsection{Local Binary Patterns (LBP)}
Another feature vector that was used was Local Binary Patterns or LBP. It was
computed by resizing each image to e.g. 100x100 pixels and then converting it
into grayscale and dividing it into uniform cells, for example 10x10 pixels for
each cell.

And then for each cell; go through each pixel and calculate a number. The
number is calculated by comparing the 8 neighbouring pixels clockwise to the
pixel value. If the neighbour is larger or equal to the pixel then a 1 is
written to the number, otherwise a zero. The final number is then interpreted
as an 8-bit binary number. When the number for all pixels in the cell has been
computed, a normalized histogram vector is created from the distribution of
numbers within the cell. This is then performed for each cell and the final
result is a concatenation of all created histograms.

To achieve rotation invariance, the histogram bins were changed such that all
non-uniform numbers end up in the same bin, while each uniform number has its
own bin. Uniform numbers are numbers where the binary number transitions from 0
to 1 or 1 to 0 at most 2 times. For example "01000000" is uniform while
"01010101" is not.

\subsubsection{Speeded up robust features (SURF)}
% TODO

\subsubsection{AlexNet Convolutional Neural Network (CNN)}
% TODO


\subsection{Training and classification}
% TODO

\subsubsection{Support Vector Machine (SVM)}
% TODO

\subsubsection{Self-organizing map (SOM)}
Self-organizing maps were considered as they can provide a good visualization
of the classifier which can help to analyze and improve the classification.

Classification with SOM was performed by creating and training a
two-dimensional map using the training dataset according to the SOM algorithm.
As the SOM is typically an unsupervised algorithm the labels are not used while
training the map. The labels are instead used afterwards to determine what tile
is considered what class.

Each tile of the map is considered to belong to the class that has most
classified samples for that tile. When classifying a new sample, it is placed
into a tile by using the SOM and then classified as the class of which that
tile belongs to.

The SOM has one parameter that was adjusted; the dimensions of the map. As the
problem has 8 classes the map must have at least 8 tiles to be able to
differentiate between all classes. But each class could also have more than one
tile as each class may have a set of subclasses. The dimensions was tested and
set as low as possible without losing accuracy as the higher dimension the
higher the training time is.
    
\end{document}
